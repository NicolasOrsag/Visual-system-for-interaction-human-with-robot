\chapter{Discusion}\label{chap:discusion}
V kapitole 2 sme otestovali rýchlosť a presnosť pôvodného algoritmu. Z týchto hodnôt sme následne vychádzali pri porovnaní s našim vylepšením tohto algoritmu. Pri 1-shot detekcii sme dosiahli presnost novel mAP 9.41, tréning trval 55 min. a tréning sme mohli vykonávať maximálne pre batch size 2, kvôli pamäťovej náročnosti na GPU. Zistili sme taktiež, že čas tréningu nám lineárne rastie pri zvyšovaní počtu trénovacích obrázkov, avšak nárast presnosti nie je lineárny a medzi 10-shot a 15-shot detekciou je minimálny rozdiel v presnosti.  

V kapitole 3 sme robili všetky pokusy pri 1-shot detekcii. Spravili sme niekoľko neúspešných pokusov o zrýchlenie. Nakoniec sa nám podarilo tréning  zrýchliliť tým, že sme vynechali všetky zamrazené vrstvy a uložili sme si výstup priamo z box head pre každý obrázok vo všetkýh jeho rozmeroch po augmentácii do súborov na disk. Náš tréning trval iba necelých 7 min. a naša presnosť sa zachovala novel mAP 10.218. Ukladanie týchto výstupov na disk nám znížilo pamäťovú náročnosť pre GPU a umožnilo nám trénovať na rôznych batch size. Vyskúšali sme otestovať presnosť a rýchlosť nášho algoritmu pri rôznych batch size. Najlepšie výsledky dosiahol pri batch size 2. 

V kapitole 4 sme otestovali náš algoritmus pri rôznom počte novel tried, pri zvyšovaní počtu tried sa znižovala aj presnosť. Vyskúšali sme taktiež trénovať iba na novel triedach, avšak nemalo to očakávaný efekt a naša presnosť sa nezvýšila.

V kapitole 5 sme vyskúšali testovať náš algoritmus na snímkoch z kamery robota NICO. Dosiahli sme presnosť mAP 14.545 pri 10-shot detekcii mandarinky. Pokúsili sme sa zo snímkov odstrániť skreslenie, to však zredukovalo aj počet našich snímkov na ktorých bola mandarinka a naša presnosť sa znížila na mAP 2.545. Naša presnosť na snímkoch z kamery robota NICO bola oveľa nižšia ako presnosť v predošlých testoch. Myslím, že za to môže hlavne kvalita a nízky počet snímkov pre testovaie.