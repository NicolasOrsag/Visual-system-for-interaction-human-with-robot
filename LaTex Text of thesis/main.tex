\documentclass[12pt, a4paper, oneside]{book}
\usepackage[hidelinks]{hyperref}
\usepackage[slovak]{babel}
\usepackage{epsfig}
\usepackage{epstopdf}
\usepackage[chapter]{algorithm}
\usepackage{algorithmic}
\usepackage{listings}
\usepackage{amsmath}
\usepackage{amssymb}
\usepackage{graphicx}
\usepackage{multirow}
\usepackage{color}
\usepackage{url}
\usepackage[utf8]{inputenc}
\usepackage[T1]{fontenc}
\usepackage{setspace}
\usepackage{tabularx}
\usepackage{textcomp}
\usepackage{caption}
\usepackage{natbib}

\setstretch{1.5}
%\renewcommand\baselinestretch{1.5} % riadkovanie jeden a pol

% pekne pokope definujeme potrebne udaje
\newcommand\mftitle{Vizuálny systém pre interakciu ľudského učiteľa s humanoidným robotom}
\newcommand\mfthesistype{Diplomová práca}
\newcommand\mfauthor{Bc. Nicolas Orság}
\newcommand\mfadvisor{Ing. Viktor Kocur, PhD.}
\newcommand\mfplacedate{Bratislava, 2022}
\newcommand\mfuniversity{UNIVERZITA KOMENSKÉHO V BRATISLAVE}
\newcommand\mffaculty{FAKULTA MATEMATIKY, FYZIKY A INFORMATIKY}
\newcommand{\sub}[1]{$_{\text{#1}}$}
\newcommand{\reference}[1]{č.~\ref{#1}}
\newcommand{\imageHeight}{150px}

\ifx\pdfoutput\undefined\relax\else\pdfinfo{ /Title (\mftitle) /Author (\mfauthor) /Creator (PDFLaTeX) } \fi

\begin{document}

\frontmatter

\thispagestyle{empty}

\noindent
\begin{minipage}{\textwidth}
\begin{center}
\textbf{\mfuniversity \\
\mffaculty}
\end{center}
\end{minipage}

\vfill
\begin{figure}[!hbt]
	\begin{center}
		\includegraphics{images/logo_fmph}
		\label{img:logo}
	\end{center}
\end{figure}
\begin{center}
	\begin{minipage}{0.8\textwidth}
		\center{\textbf{\Large\MakeUppercase{\mftitle}}}
		\smallskip
		\centerline{\mfthesistype}
	\end{minipage}
\end{center}
\vfill
2022 \hfill
\mfauthor
\eject 
% koniec obalu

\thispagestyle{empty}

\noindent
\begin{minipage}{\textwidth}
\begin{center}
\textbf{\mfuniversity \\
\mffaculty}
\end{center}
\end{minipage}

\vfill
\begin{figure}[!hbt]
\begin{center}
\includegraphics{images/logo_fmph_dark}
\label{img:logo_dark}
\end{center}
\end{figure}
\begin{center}
\begin{minipage}{0.8\textwidth}
\center{\textbf{\Large\MakeUppercase{\mftitle}}}
\smallskip
\centerline{\mfthesistype}
\end{minipage}
\end{center}
\vfill
\begin{tabular}{l l}
%Registration number: & 40a99bd8-3cb6-4534-9330-c7fd9b5e5ca4 \\
Študijný program: & Aplikovaná informatika\\
Študijný odbor: & 2511 Aplikovaná informatika\\
Školiace pracovisko: & Katedra aplikovanej informatiky\\
Školiteľ: & \mfadvisor
\end{tabular}
\vfill
\noindent
\mfplacedate \hfill
\mfauthor
\eject 
% koniec titulneho listu

%\thispagestyle{empty}
%\includegraphics[width=\textwidth]{images/zadanie}
%\vfill
%\eject
% koniec zadania

\thispagestyle{empty}


\begin{figure}[H]
\begin{center}
\makebox[\textwidth]{\includegraphics[width=\paperwidth]{images/zadaniedp}}
\label{img:zadanie}
\end{center}
\end{figure}

{~}\vspace{12cm}

\noindent
\begin{minipage}{0.25\textwidth}~\end{minipage}
\begin{minipage}{0.75\textwidth}
Čestne prehlasujem, že túto diplomovú prácu som vypracoval samostatne len s použitím uvedenej literatúry a za pomoci konzultácií u môjho školiteľa.
\newline \newline
\end{minipage}
\vfill
~ \hfill {\hbox to 6cm{\dotfill}} \\
\mfplacedate \hfill \mfauthor
\vfill\eject 
% koniec prehlasenia

\chapter*{Poďakovanie}\label{chap:thank_you}
Rád by som vyjadril úprimnú vďaku tým, ktorí mi pomohli pri dokončení tejto diplomovej práce.

Najprv by som chcel poďakovať svojmu vedúcemu práce, Ing. Viktorovi Kocurovi, PhD., za jeho cenné rady, odborné vedenie a trpezlivosť. Vždy mi bol nápomocný a reagoval veľmi rýchlo, jeho spätnú väzbu so dostával vždy takmer okamžite. 

Rovnako by som chcel poďakovať svojim rodinným príslušníkom a priateľom za ich  podporu a povzbudenie v priebehu celého procesu tvorby tejto práce. Ich podpora bola pre mňa veľmi dôležitá.

Nakoniec by som chcel poďakovať všetkým ľuďom, ktorí mi pomohli priamo alebo nepriamo s touto prácou.

Ešte raz by som sa chcel poďakovať všetkým, ktorí mi pomohli s touto diplomovou prácou. Bez Vášho prínosu by táto práca nebola možná.

\vfill\eject 
% koniec podakovania

\chapter*{Abstrakt}\label{chap:abstract_sk}

Oblasť interakcie človeka a robota výrazne pokročila v posledných rokoch, pričom humanoidné roboty sa vyvíjajú pre široké spektrum aplikácií. Efektívna interakcia medzi týmito robotmi a ich používateľmi je však stále výzvou. Táto práca sa zameriava na few-shot algoritmy pre objektovú detekciu, ktoré používajú iba malé množstvo anotovaných dát na tréning, čím minimalizujú náklady a čas potrebné na anotáciu. Skúmame algoritmus Frustratingly Simple Few-Shot Object Detection, testujeme jeho presnosť a rýchlosť pri rôznom počte trénovacích obrázkov na datasete PASCAL VOC a pri rôznom počte nových tried, ktoré sme získali z roboflow. Výsledkom našej práce je 8-násobné zrýchlenie a výrazné zníženie pamäťovej náročnosti algoritmu na GPU, pri zachovaní rovnakej presnosti. Nakoniec sme náš zrýchlený algoritmus otestovali na snímkoch z kamery robota NICO, kde sme dosiahli presnosť AP50 18.182.

~\\
Kľúčové slová: Počítačové videnie, Humanoidný robot, Hlboké učenie, Konvolučné neurónové siete, Few-shot object detection 
\vfill\eject 

\chapter*{Abstract}\label{chap:abstract_en}

The field of human-robot interaction has made significant progress in recent years, with humanoid robots being developed for a wide range of applications. However, effective interaction between these robots and their users remains a challenge. This work focuses on few-shot algorithms for object detection, which use only a small amount of annotated data for training, thus minimizing the costs and time required for annotation. We explore the Frustratingly Simple Few-Shot Object Detection algorithm, test its accuracy and speed with varying numbers of training images on the PASCAL VOC dataset and with varying numbers of new classes obtained from roboflow. The result of our work is an 8-fold speedup and significant reduction in the algorithm's memory requirements on the GPU, while maintaining the same accuracy. Finally, we tested our accelerated algorithm on images from the NICO robot's camera, achieving an AP50 accuracy of 18.182.

~\\
Keywords: Computer vision, Humanoid robot, Deep learning, Convolutional neural networks, Few-shot object detection 
\vfill\eject 
% koniec abstraktov

\tableofcontents

\mainmatter

% treba este prejst dokument ci je kod spravne formatovany
\input 01intro.tex
% \input 02motivation.tex
\input 03issues_overview.tex
\input 04testing_fsfsodt.tex
\input 05making_algorithm_faster.tex
\input 06testing_more_classes.tex
\input 07testing_on_NICO.tex
\input 08discusion.tex
\input 09conclusion.tex

\backmatter

\nocite{*}
\bibliographystyle{unsrt}
\bibliography{references}

\listoffigures

\end{document}