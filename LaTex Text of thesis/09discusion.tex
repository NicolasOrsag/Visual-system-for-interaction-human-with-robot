\chapter*{Discusion}\label{chap:discusion}
V kapitole 2 sme otestovali rýchlosť a presnosť pôvodného algoritmu. Z týchto hodnôt sme následne vychádzali pri porovnaní s našim vylepšením tohto algoritmu. Pri 1-shot detekcii sme dosiahli presnost novel mAP = 9.41, tréning trval 55 min. a tréning sme mohli vykonávať maximálne pre batch size = 2, kvôli pamäťovej náročnosti na GPU. Zistili sme taktiež, že čas tréningu nám lineárne rastie pri zvyšovaní počtu trénovacích obrázkov, avšak nárast presnosti nie je lineárny a medzi 10-shot a 15-shot detekciou je minimálny rozdiel v presnosti.  


V kapitole 3 sme robili všetky pokusy pri 1-shot detekcii. Pokúsili sme sa tento algoritmus zrýchliť najprv pomocou zníženia počtu iterácií tréningu, čo bolo neúspešné a mali sme signifikantné zníženie presnosti. Následne sme sa pokúsili pridať padding pre všetky obrázky tak aby mali rozmery najväčšieho obrázku, aby sa po pridaní do batchu nemenil ich rozmer a zapamätať si výstup z backbone pre každý obrázok, aby nám stačilo prejsť backbonom len raz, avšak obrázky boli príliš veľké a nestačila nám na to grafická karta. Potom sme vyskúšali resizovať obrázky na rovnakú veľkosť to však nebolo príliš úspešné a naša presnosť bola nízka v porovnaní s pôvodným algoritmom. Následne sme vyskúšali použiť bach size = 1 a vynechať augmentácie a tak sme si zapamätali výstup z backbone pre každý obrázok, dĺžku tréningu sa nám sice podarilo znížiť na 20 minút avšak presnosť taktiež veľmi klesla na novel mAP = 1.52. Tak sme si skúsili zapamätať príznaky pre každý obrázok vo všetkých jeho rozmeroch, uložením do súboru. Tento pokus bol veľmi úspešný a dosiahli sme presnosť novel mAP = 10.262, pri čase tréningu 33 min. Následne sme tréning ešte zrýchlili tým, že sme vynechali aj ďalšie zamrazené vrstvy a uložili sme si výstup priamo z box head, náš tréning trval iba necelých 7 min. a naša presnosť sa zachovala novel mAP = 10.218. Následne sme upravili implementáciu aby fungovala pre rôznu batch size, keďže sem ukladali výstupy do súborov na disk nezaťažovali pamäť našej GPU. Vyskúšali otestovať presnosť a rýchlosť nášho algoritmu pri rôznych batch size. Najlepšie výsledky dosiahol pri batch size = 2. 

V kapitole 4 sme otestovali náš algoritmus pri rôznom počte novel tried, pri zvyšovaní počtu tried sa znižovala aj presnosť. Vyskúšali sme taktiež trénovať iba na novel triedach, avšak nemalo to očakávaný efekt a naša presnosť sa nezvýšila.

V kapitole 5 sme vyskúšali testovať náš algoritmus na snímkoch z kamery robota NICO. Dosiahli sme presnosť mAP = 14.545 pri 10-shot detekcii mandarinky. Pokúsili sme sa zo snímkov odstrániť skreslenie, to však zredukovalo aj počet našich snímkov na ktorých bola mandarinka a naša presnosť sa znížila na mAP = 2.545. Naša presnosť na snímkoch z kamery robota NICO bola oveľa nižšia ako presnosť v predošlých testoch. Myslím, že za to môže hlavne kvalita a nízky počet snímkov pre testovaie.