\chapter*{Záver}\label{chap:conclusion}
V našej práci sme popísali základné pojmy a princípy v počítačovom videní, popísali sme metódy hlbokého učenia a few-shot objektovej detekcie. Zamerali sme sa na metódu Frustratingly Simple Few Shot Object Detection~\cite{FSFSODT}. 

Spravili sme na tomto algoritme experimenty pre rôzny počet trénovacích obrázkov. Podarilo sa nám tento algoritmus 8 násobne zrýchliť, znížiť jeho pamäťovú náročnosť na GPU a udržať rovnakú presnosť. Okrem toho sme otestovali presnosť tohto algoritmu pri zvyšujúcom sa počte novel tried a zistili sme, že si zachováva celkom solidnú presnosť. Nakoniec sme použili tento algoritmus na tréning a vyhodnotenie modelu pomocou snímkov z kamery lavého oka robota NICO.

Ako pokračovanie výskumu do budúcna by som videl testovanie algoritmu na viacerých kvalitných a rôznorodých snímkoch z kamery robota NICO a testovanie detekcie viacerých objektov. Práca môže byť taktiež rozšírená o použitie iných metód deep learningu pre few-shot objektovú detekciu, alebo použitie iných datasetov a vstupov. Veríme, že naše výsledky a popisované postupy môžu slúžiť ako užitočný nástroj pre ďalší výskum v oblasti few-shot objektovej detekcie.

