\chapter{Úvod do problematiky}\label{chap:issues_overview}

V tejto úvodnej kapitole si popíšeme a vysvetlíme základné pojmy, ktorých znalosť je nevyhnutná v našej práci. Vysvetlíme si ako funguje počítačové videnie, objektová detekcia, konvolučné neuronové siete a taktiež sa zameriame na výskum objektovej detekcie pri malej trénovacej množine (Few shot object detection), ktorý budeme chcieť využiť v našej práci pre učenie nových objektov aj z veľmi malého množstva dát.

\section{Počítačové videnie}


\setlength{\parindent}{20pt}

\hspace{\parindent}Počítačové videnie je súčasnej dobe veľmi rastúci a progresívny smer v informatike. Snaží sa priblížiť vnímaniu sveta z pohľadu ľudského oka, ktoré je pre nás prirodzené a automaticky sme schopný rozpoznávať objekty, farby a kontext toho čo vidíme. Avšak plné sémanticke pochopenie videnej reality je veľmi komplexné a zatiaľ nie sme schopný ho získať spracovaním digitálneho obrazu. Hlavne preto, že pochopenie obrazu môže vyplývať zo súvislostí, ktoré nie sú súčasťou obrazu.

Avšak počítačové videnie sa posúva veľmi rýchlo vpred. Neustále vynikajú nové algoritmy a prístupy či už na detekciu objektov alebo klasifikáciu obrazu. Medzi základné problémy počítačového videnia patrí klasifikácia, objektová detekcia a segmentácia. Pri klasifikácii sa snažíme obraz priradiť do jednej z tried. V objektovej detekcii sa snažíme v obraze určiť oblasti všetkých známych objektov a priradiť ich do tried. A pri segmentácii je našim cieľom rozdeliť obraz do viacero oblastí, každému pixlu určiť oblasť do ktorej patrí. 

\section{Príznaky}

\hspace{\parindent}Pri riešení problémov v počítačovom videní sa využívajú príznaky. Príznak v počítačovom videní je meratelný kus dát v obrázku, ktorý je unikátny pre špecifický objekt. Príznak môže reprezentovať napríklad štýl sfarbenia, nejaký tvar, či už čiaru alebo hranu v obraze alebo nejakú časť obrazu. Vďaka dobrému príznaku dokážeme od seba rozlíšiť objekty. Napríklad ak máme rozlíšiť mačku a bicykel tak ako dobrý príznak by mohlo byť, že na obrázku sa nachádza koleso. Hneď by sme vedeli vďaka tomuto príznaku klasifikovať obrázok do týchto dvoch tried. Ak by sme však mali za úlohu zistiť či je na obrázku motorka alebo bicykel, tak by nám tento príznak veľmi nepomohol a museli by sme pozerať na iné príznaky. Preto zväčša neextrahujeme z obrázku len jeden príznak, ale pre lepšiu detekciu vyberáme viacej príznakov, ktoré tvoria príznakový vektor. 

Nie je presná definícia aké príznaky obrázku by sme mali použiť, ale závisí to skôr od nášho cieľu a typu úlohy. Príznaky sa delia na lokálne a globálne. Globálne príznaky sú také, ktoré platia pre celý obrázok. Napríklad ako ako veľmi sú dominantné jednotlivé farby v obrázku. Globálny príznak nám opisuje obraz ako celok a mal by reporezentovať nejakú jeho špecifickú vlastnosť. Lokálne príznaky sa extrahujú len z určitej zaujímavej oblasti v obrázku, využivajú sa najmä pri objektovej detekcii. Najskôr nájdeme zaujímavé oblasti, ktoré by mohli reprezentovať nejakú zaujímavú vlastnosť alebo nejaký objekt. Následne vytvoríme príznakový vektor pre danú oblasť, ktorý by nám mal poskytnúť zásadnú informáciu o tejto časti obrazu. Treba rátať s tým, že objekt na obrázku môže byť rôznej veľkosti, rôzne natočený, rôzne osvetlený, zašumený, môže sa nachádzať v rôznych častiach obrázku a podobne. Preto naše príznaky by mali byť čo najviac odolné voči týmto zmenám(mali by ich čo najmenej ovplyvňovať). Čím viac invariatné príznaky voči týmto zmenám si zvolíme tým lepšia je naša detekcia. 

\section{Objektová detekcia}

\hspace{\parindent}Asi najskúmanejším problémom v počítačovom videní je objektová detekcia, ktorá spočíva v rozpoznaní jednotlivých objektov a ich pozícii v digitálnom obraze. Dá sa k tomuto problému pristupovať rôznymi spôsobmi, dá sa pristupovať k tomuto problému tradičnými metódami počítačového videnia, alebo dnes už veľmi rozšireným s oveľa lepšími a presnejšími výsledkami, ako pri tradičných metódach a to pomocou hlbokého učenia, ktorých klúčom je naučiť sa na veľkých dátach extrahovať príznaky tak aby mala detekcia čo najväčšiu presnosť.

\subsection{Tradičé metódy}
\hspace{\parindent}Tradičné metódy v objektovej detekcii majú zvyčajne tri etapy: vybratie oblasti, extrakcia príznakov, klasifikácia objektu. 

V prvej etape sa snažíme lokalizovať objekt. Keďže objekt môže byť rôznej veľkosti, musíme skenovať celý obrázok pomocou posúvneho okna rôznej veľkosti. Táto metóda je výpočtovo náročná. 

V druhej etape použijeme použijeme metódy ako SIFT, HOG na extrakciu vizuálnych príznakov na rozpoznanie objektu. Tieto  príznaky 

