\chapter*{Úvod}\label{chap:intro}
Detekcia objektov je kľúčovým úlohou v oblasti počítačového videnia s rôznymi aplikáciami vo vybraných oblastiach, ako sú robotika, dozor a autonomné riadenie. Tradičné algoritmy detekcie objektov spočívajú v potrebe veľkého množstva anotovaných trénovacích vzoriek na dosiahnutie vysokého stupňa presnosti. Získanie takýchto datasetov môže byť náročné a nákladné, najmä v prípadoch, kde sa objekty záujmu vyskytujú zriedkavo alebo sú jedinečné. V posledných rokoch sa ako sľubný prístup na riešenie tohto problému ukázalo učenie s malým množstvom vzoriek (few-shot learning).

Few-shot learning sa snaží umožniť učenie nových konceptov s obmedzeným množstvom anotovaných dát. V tejto práci sa zameriavame na few-shot detekciu objektov, ktorá zahŕňa detekciu objektov záujmu pomocou len niekoľkých anotovaných obrázkoch. Naším cieľom je preskúmať few-shot algoritmy objektovej detekcie a vybrať taký, ktorý bude možno trénovať efektívne. Následne otestujeme jeho rýchlosť a presnosť pre rôzny počet trénovacích obrázkov a rôzny počet detegovaných tried. Pokúsime sa ho zefektívniť pre naše potreby, natrénovať a otestovať model pre few-shot objektovú detekciu na záberoch z kamery humanoidného robota NICO. 

V kapitole 1 si vysvetlíme základné pojmy a princípy počítačového videnia a objektovej detekcie. V kapitole 2 otestujeme presnosť a rýchlosť algoritmu Frustratingly Simple Few-Shot Object Detection~\cite{FSFSODT} po pridaní novej triedy pri rôznom počte trénovacích obrázkov. V kapitole 3 spravíme niekoľko pokusov s cieľom zrýchliť tento algoritmus, čo sa nám nakoniec úspešne podarilo. V kapitole 4 otestujeme náš zrýchlený algoritmus pri rôznom počte nových tried. V poslednej kapitole 5 otestujeme náš zrýchlený algoritmus na vlastnom datasete, ktorý tvoria snímky z kamery robota NICO. 

Link na repozitár k tejto práci: \href{https://github.com/NicolasOrsag/Visual-system-for-interaction-human-with-robot}{https://github.com/NicolasOrsag/Visual-system-for-interaction-human-with-robot}

