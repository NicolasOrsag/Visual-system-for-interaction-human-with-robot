\chapter{Motivácia}\label{chap:motivation}
Predstavte si situáciu, keď humanoidný robot má za úlohu identifikovať a lokalizovať nebezpečné predmety, ako sú napríklad ostré nástroje alebo nebezpečné chemikálie, v prostredí, ktoré robot predtým nepoznal. Tradičné prístupy k objektivej detekcii vyžadujú veľké množstvo anotovaných dát na trénovanie modelu, ktorý dokáže s vysokou presnosťou rozpoznať nebezpečné predmety v novom prostredí. Získanie takýchto datasetov môže byť nákladné a časovo náročné. Preto môže byť few-shot learning užitočným nástrojom pre tento konkrétny prípad, keďže umožňuje učenie nových konceptov s obmedzeným množstvom anotovaných dát. Tým pádom sa znižuje požiadavka na veľké množstvo anotovaných dát a časová a finančná náročnosť získavania trénovacích dát.
